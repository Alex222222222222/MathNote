\subsection{Exercise}

\begin{enumerate}
      \item Give an counter example why \mt{
            \closure{\cup A_{\alpha}} = \cup \closure{A_{\alpha}}
      } dose not hold.

      \begin{proof}
            Consider the X be the K-topology on the real line.
      
            Let
            \begin{eqnarray*}
                  A_{n} &=& (\frac{1}{n+1},\frac{1}{n}), n \in \mathbb{Z}_{+} \\
                  A &=& \cup A_{n}
            \end{eqnarray*}
      
            Then 
            \begin{eqnarray*}
                  \closure{A_{n}} &=& [\frac{1}{n+1},\frac{1}{n}] \\
                  \cup \closure{A_{n}} &=& (0,1]
            \end{eqnarray*}
      
            However, as every neighbourhood of \mt{0} intersect \mt{\cup A_{\alpha}}. \mt{0 \in \closure{\cup A_{\alpha}}}.
      
            Thus, \mt{
                  \closure{\cup A_{\alpha}} \neq \cup \closure{A_{\alpha}}
            }
      \end{proof}

      \item Prove that 
      \begin{equation*}
            \closure{A-B} \supseteq \closure{A} - \closure{B}
      \end{equation*}

      \begin{proof}
            If \mt{x \in \closure{A} - \closure{B}}. Then
            \begin{eqnarray*}
                  x \in \closure{A}, x \notin \closure{B}
            \end{eqnarray*}.

            Thus for open set \mt{U} containing \mt{x}
            \begin{eqnarray*}
                  &\exists& U_{1} \cap B = \emptyset \\
                  &\forall& U \cap A \neq \emptyset
            \end{eqnarray*}

            Suppose that \mt{x \notin \closure{A-B}}. Then
            \begin{eqnarray*}
                  \exists U_{0} \cap (A-B) = \emptyset
            \end{eqnarray*}

            Thus,
            \begin{eqnarray*}
                  U_{0} \cap A \subseteq B
            \end{eqnarray*}

            Thus,
            \begin{eqnarray*}
                  U_{1} \cap B &=& \emptyset \\
                  U_{1} \cap U_{0} \cap A &=& \emptyset
            \end{eqnarray*}

            As \mt{U_{1} \cap U_{0}} is an open set containing \mt{x}, so there is contradiction with \mt{x \in \closure{A}}. Thus \mt{x \in \closure{A-B}}.
      \end{proof}

      \item A \defineNewWord{diagonal}\label{def:Diagonal} is a subset \mt{
            \Delta = \{ x \times x | x \in \mathbb{X} \}
      } of the product topology \mtb{X \times X} where \mtb{X} is a topological space. Show that the diagonal is closed in \mtb{X \times X} \ioi \mtb{X} is a Hausdorff space.

      \begin{proof}
            If \mtb{X} is a Hausdorff space. For every element \productSet{x}{y} of \productSet{\mathbb{X}}{\mathbb{X}} that not in \mt{\Delta}. We take disjoint set \mt{U_{x},U_{y}} where \mt{
                  x \in U_{x}, y \in U_{y}
            }. Then \mt{
                  \mathbb{X} \times \mathbb{X} - \Delta = \cup_{x \neq y} U_{x} \times U_{y}
            }. Where \mt{\cup_{x \neq y} U_{x} \times U_{y}} is an open set. Thus \mt{\Delta} is a closed set.

            Conversely, if \mt{\Delta} is a closed set, suppose that \mtb{X} is not a Hausdorff space. Then there exists distinct \mt{x, y} such that every neighbourhood of \mt{x} and \mt{y} intersect. Let \mtb{B} be a basis of topology of \mtb{X}. Then \mt{
                  x \times y \in \mathbb{X} \times \mathbb{X} - \Delta
            }. However we cannot find \mt{
                  B_{1}, B_{2} \in \mathbb{B}, x \times y \in B_{1} \times B_{2} \subset \mathbb{X} \times \mathbb{X} - \Delta
            }. Then \mt{\Delta} is not a closed set. So there is a contradiction, then \mtb{X} must be a Hausdorff space.
      \end{proof}

      \item Prove that \mt{T_{1}} axiom is equivalent to the condition such that for every distinct pair \mt{x,y} of \mtb{X}, there exists neighbourhood of \mt{x} does not contain \mt{y}.
      
      \begin{proof}
            First if \mt{T_{1}} axiom hold, then for every pair \mt{x,y}, the neighbourhood \mt{\mathbb{X}-\{y\}} of \mt{x} does not contain \mt{y}, so the second condition hold.

            Conversely, if the second condition hold. Suppose that we can find a finite points set say \mt{\{x_{1}, x_{2}, x_{3} \dots}\}, then there must exists \mt{x \in \{x_{1}, x_{2}, x_{3} \dots}\} such that the set \mt{\{x\}} is not closed. Then \mt{
                  \closure{\{x\}} - \{x\} \neq \emptyset
            }. Let \mt{y \in \closure{\{x\}} - \{x\} }, then every neighbourhood of y must contain \mt{x}, this is a contradiction to the second condition, so the \mt{T_{1}} axiom must hold.
      \end{proof}

      \item If \mt{A \subseteq \mathbb{X}}, we define the \defineNewWord{boundary}\label{def:Boundary} of \mt{A} by the equation
      \begin{equation*}
            \text{Bd} A = \closure{A} \cap \closure{\mathbb{X}-A}
      \end{equation*}
      \begin{enumerate}
            \item Show that \mt{\text{Int}A} and \mt{\text{Bd} A} are disjoint and \mt{
                  \closure{A} = \text{Int} A \cup \text{Bd} A
            }.
            
            \begin{proof}
                  For every \mt{ x \in \boundary{A} }, every open set contain \mt{x} must intersect \mt{A} and \mt{\mathbb{X}-A} so, there is no open set \mt{U} contain \mt{x}, \mt{U \subseteq A}.

                  For every \mt{x' \in \interior{A}}, there exists \mt{U' \subseteq A}, so \mt{\boundary{A}} and \mt{\interior{A}} are disjoint sets.

                  For every \mt{x \in \closure{A}}, \mt{x \in \boundary{A}} or \mt{x \notin \boundary{A}}. We discuss the condition that \mt{x \notin \boundary{A}}.

                  Then \mt{x \notin \closure{\mathbb{X}-A}}, then there exists a open set \mt{U} containing \mt{x}, that does not intersect with \mt{\mathbb{X}-A}. Thus \mt{U \subseteq A}, thus \mt{x \in \interior{A}}. So \mt{
                        \closure{A} \subseteq \text{Int} A \cup \text{Bd} A
                  }.

                  Then, \mt{\boundary{A} \subseteq \closure{A}}, \mt{\interior{A} \subseteq A \subseteq \closure{A}}. Thus, \mt{
                        \closure{A} \supseteq \text{Int} A \cup \text{Bd} A
                  }

                  So, \mt{
                        \closure{A} = \text{Int} A \cup \text{Bd} A
                  }
            \end{proof}

            \item Show that \mt{\boundary{A} = \emptyset} \ioi \mt{A} is both open and closed.
            
            \begin{proof}
                  So, \mt{\interior{A} = \closure{A}}, then \mt{\boundary{A} = \emptyset} follows directly from \mt{
                        \closure{A} = \text{Int} A \cup \text{Bd} A
                  }.
            \end{proof}

            \item Show that \mt{U} is open \ioi \mt{\boundary{U} = \closure{U}-U}.
            
            \begin{proof}
                  Suppose U is open. Then \mt{\closure{\mathbb{X}-U} = \mathbb{X} - U}. Then for every \mt{x \in U}, \mt{x \notin \mathbb{X} - U, x \notin \closure{\mathbb{X}-U}}. Thus \mt{\closure{U} \cap \closure{\mathbb{X}-U}=\closure{U}-U}.

                  Conversely, suppose \mt{\boundary{U} = \closure{U}-U}. Then for every \mt{x \in U}, \mt{x \notin \boundary{U}}. Then as \mt{\closure{U} = \interior{U}\cup \boundary{U}}, \mt{x \in \interior{U}}. So \mt{\interior{U} \supseteq U}. Thus \mt{U = \interior{U}}. Thus, \mt{U} is open.
            \end{proof}

      \end{enumerate}
\end{enumerate}

