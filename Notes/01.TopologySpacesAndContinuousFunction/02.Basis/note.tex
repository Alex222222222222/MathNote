\section{Basis for a Topology}

      \begin{definition}[basis]\label{def:Basis}
            If $ \mathbb{X} $ is a set, a \defineNewWorld{basis} for a topology on \textmathbb{X} is a collection \textmathbb{B} of subsets of \textmathbb{X} (called \defineNewWorld{basis elements}) such that:
            \begin{itemize}
                  \item For each $ x \in \mathbb{X} $, there is at least one basis element $ B $ containing $ x $
                  \item If $ x $ belongs to the intersection of two basis elements $ B_{1} $ and $ B_{2} $, then there is another element $ x \in B_{3} \in \mathbb{B} $ such that $ B_{3} \subseteq B_{1} \cap B_{2} $
            \end{itemize}
      \end{definition}

      \begin{definition}[topology generated by basis]\label{def:TopologyGeneratedByBasis}
            Let \textmathbb{B} be a basis on \textmathbb{X}. Let \textmathbb{U} be a set containing all subsets $ U $ of \textmathbb{X} such that for each element $ x \in U $, there is $ B \in \mathbb{B} $ that $ x \in B \subseteq U $.
            Such \textmathbb{U} formed a topology on \textmathbb{X}, called \defineNewWorld{topology \textmathbb{T} generated by \textmathbb{B}}
      \end{definition}
      
      \begin{lemma}
            Let \textmathbb{X} be a set. Let \textmathbb{B} be a basis for a topology \textmathbb{T} on \textmathbb{X}. Then \textmathbb{T} equals to the set of all possible unions of elements of \textmathbb{B}.
      \end{lemma}

      \begin{proof}
            Let set \textmathbb{U} be the set of all possible unions of elements of \textmathbb{B}. For any $ U \in \mathbb{U} $. $ U = \cup B $ \footnote{
                  Note that this expression may not be unique.
            } for some $ B \in \mathbb{B} $. Thus, for every $ x \in U $, there exist a $ B' \in \mathbb{B} $ that $ x \in B' \subseteq U $. Thus, $ U \in \mathbb{T} $.

            Conversely, for any $ U \in \mathbb{T} $. For any $ x \in U $, let $ x \in B_{x} \in U $. Then, $ U = \cup_{x \in U}B_{x} $. Thus, $ U \in \mathbb{U} $.

            Therefore, \textmathbb{U} equals to \textmathbb{T}.
      \end{proof}

      \begin{lemma} \footnote{We omit the proof of this lemma as it is obvious.}
            Let \textmathbb{X} be a topological space. Suppose that \textmathbb{C} is a collection of open sets of \textmathbb{X} such that for each open set $ U $ of \textmathbb{X} and each $ x \in U $, there is an element $ C \in \mathbb{C} $ such that $ x \in C \subseteq C $. Then \textmathbb{C} is a basis for the topology of \textmathbb{X}.
      \end{lemma}

      \begin{lemma} \footnote{We omit the proof of this lemma as it is obvious.}
            Let \textmathbb{B} and \textmathbb{B'} be basis for the topologies \textmathbb{T} and \textmathbb{T'}, respectively, on \textmathbb{X}. Then the following are equivalent:
            \begin{itemize}
                  \item \textmathbb{T'} is finer than \textmathbb{T}
                  \item For each $ x \in \mathbb{X} $ and each basis element $ B \in \mathbb{B} $ containing X, there is a basis element $ B' \in \mathbb{B'} $ such that $ x \in B' \subseteq B $.
            \end{itemize}
      \end{lemma}

      \begin{definition}[standard topology on the real line]\label{def:StandardTopologyOnTheRealLine}
            Let be $ \mathbb{B} = \{ B | B = \{ x | a < x < b \}, a < b, a \in \mathbb{R}, b \in \mathbb{R} \} $. \textmathbb{B} formed a basis on real line. The topology generated by \textmathbb{B} is called the \defineNewWorld{standard topology on the real line} \footnote{
                  Whenever we consider \textmathbb{R}, we shall suppose it is given this topology unless we specifically state otherwise.
            } .
      \end{definition}

      \begin{definition}[lower limit topology on the real line]\label{def:LowerLimitTopologyOnTheRealLine}
            Let be $ \mathbb{B} = \{ B | B = \{ x | a \le x < b \}, a < b, a \in \mathbb{R}, b \in \mathbb{R} \} $. \textmathbb{B} formed a basis on real line. The topology generated by \textmathbb{B} is called the \defineNewWorld{lower limit topology on the real line}. When \textmathbb{R} is given this topology,we denote it by $ \mathbb{R}_{l} $.
      \end{definition}

      \begin{definition}[K-topology on the real line]\label{def:KTopologyOnTheRealLine}
            Let be $ \mathbb{B} = \{ B | B = \{ x | a < x < b \}, a < b, a \in \mathbb{R}, b \in \mathbb{R} \} $. Let $ K = \{ x | x = \frac{1}{n}, n \in \mathbb{Z_{+}} \} $. $ \mathbb{B} \cup \{ B - K | B \in \mathbb{B} \} $ formed a basis on real line. The topology generated by \textmathbb{B} is called the \defineNewWorld{K-topology on the real line}. When \textmathbb{R} is given this topology,we denote it by $ \mathbb{R_{K}} $.
      \end{definition}

      \begin{lemma} \omitObviuos
            The topologies $ \mathbb{R}_{l} $ and \textmathbb{R_{K}} is strictly finer than the standard topology on \textmathbb{R}.
      \end{lemma}

      \begin{lemma}
            The topologies of $ \mathbb{R}_{l} $ and \textmathbb{R_{K}} is not comparable.
      \end{lemma}

      \begin{proof}
            Let $ \mathbb{T}_{l} $ and \textmathbb{T_{K}} be topologies of $ \mathbb{R}_{l} $ and \textmathbb{R_{K}} respectively. Let $ K = \{ x | x = \frac{1}{n}, n \in \mathbb{Z_{+}} \} $.

            We first proof that $ \mathbb{T}_{l} $ is not finer than \textmathbb{T_{K}}. Let $ U = \{ x | -1 < x < 1 \} - K, x = 0 $.
            If there exist $ B = \{ x | a \le x < b \} \in \mathbb{T}_{l} $ such that $ x \in B \subseteq U $, then $ 0 < b < 1 $. Thus, there exist $ n \in \mathbb{Z_{+}} $ that $ 0 < \frac{1}{n} < b $. Thus $ B $ is not a subset of $ U $.

            Then we proof that \textmathbb{T_{K}} is not finer than $ \mathbb{T}_{l} $. Let $ U' = \{ x | a' \le x < b' \} $.
            If there exist $ B' = \{ x | a'' < x < b'' \} or \{ x | a'' < x < b'' \} - K $ such that $ {a'} \in B \subseteq U $.
            Thus $ a'' < a < b'' $.
            Thus there exist $ c $ that $ a'' < x < a, x \in B ,x \notin U' $. Thus $ B' \nsubseteq U' $.

            Thus the topologies of $ \mathbb{R}_{l} $ and \textmathbb{R_{K}} is not comparable.
      \end{proof}

      \begin{definition}[subbasis]\label{def:Subbasis}
            A \defineNewWorld{subbasis} \textmathbb{S} for a topology on \textmathbb{X} is a collection of subsets of \textmathbb{X} whose union equals \textmathbb{X}. The \defineNewWorld{topology generated by the subbasis} \textmathbb{S} is defined to be the collection \textmathbb{T} \footnote{
                  It is obvious that \textmathbb{T} is a topology, we just omit the proof here.
            } of all unions of finite intersections of elements of \textmathbb{S}.
      \end{definition}
