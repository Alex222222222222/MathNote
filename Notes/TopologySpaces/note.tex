\documentclass[twoside]{article}

\usepackage[a4paper, marginparwidth=75pt]{geometry}

\usepackage{amssymb}
\usepackage{amsmath}
\usepackage{amsthm}

\usepackage{marginnote}

\newtheorem{definition}{Definition}[section]
\newtheorem{lemma}{Lemma}[section]

\title{Topology}
\author{Alex}

\newcommand{\textmathbb}[1]{$ \mathbb{#1} $}
\newcommand{\defineNewWorld}[1]{\textit{\textbf{#1}}}
\newcommand{\omitObviuos}{\footnote{We omit the proof of this lemma as it is obvious.}}

\begin{document}

      \tableofcontents

      \section{g}

            \begin{definition}[topology]
                  A \textit{\textbf{topology}} on a set $ \mathbb{X} $ is a collection $ \mathbb{T} $ of subsets of $ \mathbb{X} $ having the following properties:

                  \begin{itemize}
                        \item $ \emptyset $ and $ \mathbb{X} $ are in $ \mathbb{T} $
                        \item The union of the elements of any sub collection of $ \mathbb{T} $ is in $ \mathbb{T} $
                        \item The intersection of the elements of any \textbf{finite} sub collection of $ \mathbb{T} $ is in $ \mathbb{T} $
                  \end{itemize} 
            \end{definition}

            \begin{definition}[topology space]
                  A \textit{\textbf{topological space}} is a set $ \mathbb{X} $ for which a topology $ \mathbb{T} $ has been specified.
            \end{definition}
            
            \begin{definition}[open set]
                  A \textit{\textbf{open set}} $ \mathbb{U} $ is a subset of $ \mathbb{X} $ that belongs to a topology $ \mathbb{T} $ of $ \mathbb{X} $.
            \end{definition}
            
            \begin{definition}[open sets]
                  A topology can also be called a \textit{\textbf{open sets}}
            \end{definition}

            \begin{definition}[discrete topology]
                  The set of all subsets of a set $ \mathbb{X} $ formed a topology called \textit{\textbf{discrete topology}}
            \end{definition}

            \begin{definition}[trivial topology]
                  The set consisting the set $ \mathbb{X} $ and $ \emptyset $ only formed a topology of $ \mathbb{X} $ called \textit{\textbf{trivial topology}}
            \end{definition}

            \begin{definition}[finite complement topology]
                  Let $ \mathbb{X} $ be a set. Let $ \mathbb{T}_{\mathit{f}} $ be the collection of all subsets $ \mathbb{U} $ of $ \mathbb{X} $ such that $ \mathbb{X} - \mathbb{U} $ either if a \textbf{finite} \marginpar{
                        The set $ \mathbb{U} $ can form a topology because of the definition of topology is intersection of finite sub collection. If this can be intersection of infinite sub collection, $ \mathbb{U} $ will not be a topology. 
                  } of is all of $ \mathbb{X} $. Then $ \mathbb{T}_{\mathit{f}} $ is a topology on $ \mathbb{X} $, called the \textit{\textbf{}}.
            \end{definition}

            \begin{definition}[finer, larger, strictly finer, strictly larger, coarser, smaller, strictly coarser, strictly smaller, comparable]
                  Let $ \mathbb{T} $ and $ \mathbb{T'} $ be two topology on a given set $ \mathbb{X} $. If $ \mathbb{T} $ is  a subset of $ \mathbb{T'} $, we say that $ \mathbb{T'} $ is \textit{\textbf{finer}} or \textit{\textbf{larger}} than $ \mathbb{T} $. If $ \mathbb{T} $ is a proper subset of $ \mathbb{T'} $, we say that $ \mathbb{T'} $ is \textit{\textbf{strictly finer}} or \textit{\textbf{strictly larger}}  than $ \mathbb{T} $.
                  We also say that $ \mathbb{T} $ is \textit{\textbf{coarser}} or \textit{\textbf{smaller}} or \textit{\textbf{strictly coarser}} or \textit{\textbf{strictly smaller}} than $ \mathbb{T'} $.
                  We say that $ \mathbb{T} $ and $ \mathbb{T'} $ is \textit{\textbf{comparable}} if either $ \mathbb{T} $ is a subset of $ \mathbb{T'} $ or $ \mathbb{T'} $ is a subset of $ \mathbb{T} $.
            \end{definition}



\end{document}