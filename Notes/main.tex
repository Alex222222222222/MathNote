\documentclass[twoside]{book}

\usepackage[a4paper]{geometry}

\usepackage{amssymb}
\usepackage{amsmath}
\usepackage{amsthm}
\usepackage{amsfonts}

\usepackage{multicol}

% see https://tex.stackexchange.com/a/39755
% Simulated package file
\begin{filecontents}{envcode.sty}
      \newcommand\NewEnvCode[2]{%
       \expandafter\def\csname code@#1\endcsname{#2}%
       \expandafter\def\csname change@code@#1\endcsname{%
        \expandafter\let\expandafter\Code\csname code@#1\endcsname
       }%
      }
      
      \newcommand\MakeDefault{%
       \expandafter\let\expandafter\code@@default\csname code@\@currenvir\endcsname
      }
      
      \newcommand\RunEnvCode{%
       \let\Code=\code@@default
       \csname change@code@\@currenvir\endcsname
       \Code
      }
      
      \AtBeginDocument{\MakeDefault}
\end{filecontents}

\usepackage{envcode}

% for cross reference
% \usepackage{hyperref}

% for fixed length of cell of tables
% \usepackage{tabularx}
% \usepackage{longtable}

\newtheorem{definition}{Definition}[section]
\newtheorem{lemma}{Lemma}[section]
\newtheorem{theorem}{Theorem}[section]
\newtheorem{corollary}{Corollary}[section]
\newtheorem{proposition}{Proposition}[section]

\NewEnvCode{document}{default code}
\NewEnvCode{theorem}{ theorem }
\NewEnvCode{lemma}{ lemma }
\NewEnvCode{corollary}{ corollary }
\NewEnvCode{proposition}{ proposition }
\NewEnvCode{definition}{ definition }

\title{Topology Note}
\author{Alex}

\newcommand{\textmathbb}[1]{ $ \mathbb{#1} $ }
\newcommand{\defineNewWord}[1]{\textit{\textbf{#1}}}
\newcommand{\omitObviuos}{\footnote{We omit the proof of this \RunEnvCode as it is obvious.}}
\newcommand{\mt}[1]{ $ #1 $ }
\newcommand{\tmb}[1]{\textmathbb{#1}}
\newcommand{\mtb}[1]{\textmathbb{#1}}
\newcommand{\productSet}[2]{ $ #1 \times #2 $ }
\newcommand{\closure}[1]{\overline{#1}}
\newcommand{\mclosure}[1]{ $ \overline{#1} $ }
\newcommand{\boundary}[1]{\text{Bd}#1}
\newcommand{\interior}[1]{\text{Int}#1}
\newcommand{\norm}[1]{||#1||}

% commonly used definitions
\newcommand{\topology}{topology}
\newcommand{\basis}{basis}
\newcommand{\topological}{topological}
\newcommand{\closed}{closed}
\newcommand{\hausdorff}{Hausdorff}
\newcommand{\Hausdorff}{Hausdorff}
\newcommand{\intersection}{intersection}
\newcommand{\union}{union}
\newcommand{\arbitrary}{arbitrary}
\newcommand{\intersect}{intersect}
\newcommand{\ioi}{ if and only if }
\newcommand{\element}{element}
\newcommand{\neighbourhood}{neighbourhood}
\newcommand{\converge}{converge}
\newcommand{\induction}{induction}
\newcommand{\continuous}{continuous}
\newcommand{\homeomorphism}{homeomorphism}
\newcommand{\equivalent}{equivalent}
\newcommand{\imbedding}{imbedding}
\newcommand{\diam}{\mathrm{diam}}

\begin{document}
      \maketitle

      \newpage

      \tableofcontents

      \newpage

      \section*{Definitions}

\begin{multicols}{2}

\vspace{1em}\large{\textbf{B}}

basis, \pageref{def:Basis}

boundary, \pageref{def:Boundary}

\vspace{1em}\large{\textbf{C}}

closed, \pageref{def:Closed}

\hspace{1em}closed in, \pageref{def:ClosedIn}

closure, \pageref{def:Closure}

coarser, \pageref{def:Comparable}

\hspace{1em}strictly coarser, \pageref{def:Comparable}

finer, \pageref{def:Comparable}

\hspace{1em}strictly finer, \pageref{def:Comparable}

larger, \pageref{def:Comparable}

\hspace{1em}strictly larger, \pageref{def:Comparable}

smaller, \pageref{def:Comparable}

\hspace{1em}strictly smaller, \pageref{def:Comparable}

continuous, \pageref{def:Continuous}

\hspace{1em}continuous relative to, \pageref{def:ContinuousRelativeTo}

converge, \pageref{def:Converge}

convex, \pageref{def:Convex}

coordinate functions, \pageref{def:CoordinateFunctions}

\vspace{1em}\large{\textbf{D}}

diagonal, \pageref{def:Diagonal}

discrete topology, \pageref{def:DiscreteTopology}

\vspace{1em}\large{\textbf{F}}

finite complement topology, \pageref{def:FiniteComplementTopology}

\vspace{1em}\large{\textbf{H}}

Hausdorff space, \pageref{def:HausdorffSpace}

homeomorphism, \pageref{def:Homeomorphism}

\vspace{1em}\large{\textbf{I}}

interior, \pageref{def:Interior}

intersect, \pageref{def:Intersect}

interval, \pageref{def:Interval}

\hspace{1em}closed interval, \pageref{def:Interval}

\hspace{1em}half-open interval, \pageref{def:Interval}

\hspace{1em}open interval, \pageref{def:Interval}

\vspace{1em}\large{\textbf{K}}

K-topology on R, \pageref{def:KTopologyOnTheRealLine}

\vspace{1em}\large{\textbf{L}}

limit, \pageref{def:Limit}

cluster point, \pageref{def:LimitPoint}

limit point, \pageref{def:LimitPoint}

point of accumulation, \pageref{def:LimitPoint}

lower limit topology on R, \pageref{def:LowerLimitTopologyOnTheRealLine}

\vspace{1em}\large{\textbf{N}}

neighbourhood, \pageref{def:Neighbourhood}

\vspace{1em}\large{\textbf{O}}

open map, \pageref{def:OpenMap}

open set, \pageref{def:OpenSet}

open sets, \pageref{def:OpenSets}

ordered square, \pageref{def:OrderedSquare}

order topology, \pageref{def:OrderTopology}

\vspace{1em}\large{\textbf{P}}

product topology, \pageref{def:ProductTopology}

projection, \pageref{def:Projection}

\vspace{1em}\large{\textbf{R}}

ray, \pageref{def:Ray}

\hspace{1em}closed ray, \pageref{def:Ray}

\hspace{1em}open ray, \pageref{def:Ray}

\vspace{1em}\large{\textbf{S}}

standard topology on R, \pageref{def:StandardTopologyOnTheRealLine}

subbasis, \pageref{def:Subbasis}

subspace, \pageref{def:SubspaceTopology}

subspace topology, \pageref{def:SubspaceTopology}

\vspace{1em}\large{\textbf{T}}

\mt{T_{1}} axiom, \pageref{def:T1Axiom}

topological imbedding, \pageref{def:TopologicalImbedding}

topology, \pageref{def:Topology}

topology generated by basis, \pageref{def:TopologyGeneratedByBasis}

topology space, \pageref{def:TopologySpace}

trivial topology, \pageref{def:TrivialTopology}

\end{multicols}

      \newpage

      \section*{Theorems}

\vspace{1em}\noindent\large{\textbf{M}}

Maps into products, \pageref{theorem:MapsIntoProducts}

\vspace{1em}\noindent\large{\textbf{R}}

Rules for constructing continuous functions, \pageref{theorem:RulesForConstructingContinuousFunctions}

\vspace{1em}\noindent\large{\textbf{T}}

The pasting lemma, \pageref{theorem:ThePastingLemma}



      \newpage

      \chapter{Topology Spaces and Continuous Function}

      \section{The Order Topology}

\begin{definition}[interval]\label{def:Interval}
      Let \textmathbb{X} is a set having a simple order relation $ < $. Given elements $ a $ and $ b $ of \textmathbb{X} such that $ a < b $, there are four subsets of \textmathbb{X} that are called \defineNewWorld{intervals} determined by $ a $ and $ b $:
      \begin{itemize}
            \item $ (a,b) = \{ x | a < x < b \} $
            \item $ (a,b] = \{ x | a < x \le b \} $
            \item $ [a,b) = \{ x | a \le x < b \} $
            \item $ [a,b] = \{ x | a \le x \le b \} $
      \end{itemize}

      $ (a,b) $ is called an \defineNewWorld{open interval} on \textmathbb{X}.
      $ [a,b] $ is called an \defineNewWorld{closed interval} on \textmathbb{X}.
      $ (a,b] $ and $ [a,b) $ is called \defineNewWorld{half-open intervals}.
\end{definition}

\begin{definition}[order topology]\label{def:OrderTopology}\footnote{
      The standard topology on \textmathbb{R} is an order topology derived from the usual order on \textmathbb{R}.
}
      Let \textmathbb{X} be a set with a simple order relation; assume \textmathbb{X} has more than one element. Let \textmathbb{B} be the collection of all sets of the following types:
      \begin{itemize}
            \item All open intervals $ (a,b) $ in \textmathbb{X}.
            \item All intervals of the form $ [a_{0},b) $, where $ a_0 $ is the smallest element(if exist) of \textmathbb{X}.
            \item All intervals of the form $ (a,b_{0}] $, where $ b_0 $ is the largest element(if exist) of \textmathbb{X}.
      \end{itemize}

      The collection \textmathbb{B} formed a basis for a topology on \textmathbb{X}, which is called the order topology.
\end{definition}

\begin{definition}[ray]\label{def:Ray}\footnote{
      open rays are always open sets in the order topology
}\footnote{
      the open rays also formed a subbasis of the order topology
}
      If \textmathbb{X} is an ordered set, and $ a $ is an element of \textmathbb{X}, there are four subsets of \textmathbb{X} that are called \defineNewWorld{rays} determined by $ a $:
      \begin{itemize}
            \item $ (a,+\infty) = \{ x | x > a \} $
            \item $ (-\infty,a) = \{ x | x < a \} $
            \item $ [a,+\infty) = \{ x | x \ge a \} $
            \item $ (-\infty,a] = \{ x | x \le a \} $
      \end{itemize}

      $ (a,+\infty) $ and $ (-\infty,a) $ are called \defineNewWorld{open rays}.
      $ [a,+\infty) $ and $ (-\infty,a] $ are called \defineNewWorld{closed rays}.
\end{definition}

      \section{The Order Topology}

\begin{definition}[interval]\label{def:Interval}
      Let \textmathbb{X} is a set having a simple order relation $ < $. Given elements $ a $ and $ b $ of \textmathbb{X} such that $ a < b $, there are four subsets of \textmathbb{X} that are called \defineNewWorld{intervals} determined by $ a $ and $ b $:
      \begin{itemize}
            \item $ (a,b) = \{ x | a < x < b \} $
            \item $ (a,b] = \{ x | a < x \le b \} $
            \item $ [a,b) = \{ x | a \le x < b \} $
            \item $ [a,b] = \{ x | a \le x \le b \} $
      \end{itemize}

      $ (a,b) $ is called an \defineNewWorld{open interval} on \textmathbb{X}.
      $ [a,b] $ is called an \defineNewWorld{closed interval} on \textmathbb{X}.
      $ (a,b] $ and $ [a,b) $ is called \defineNewWorld{half-open intervals}.
\end{definition}

\begin{definition}[order topology]\label{def:OrderTopology}\footnote{
      The standard topology on \textmathbb{R} is an order topology derived from the usual order on \textmathbb{R}.
}
      Let \textmathbb{X} be a set with a simple order relation; assume \textmathbb{X} has more than one element. Let \textmathbb{B} be the collection of all sets of the following types:
      \begin{itemize}
            \item All open intervals $ (a,b) $ in \textmathbb{X}.
            \item All intervals of the form $ [a_{0},b) $, where $ a_0 $ is the smallest element(if exist) of \textmathbb{X}.
            \item All intervals of the form $ (a,b_{0}] $, where $ b_0 $ is the largest element(if exist) of \textmathbb{X}.
      \end{itemize}

      The collection \textmathbb{B} formed a basis for a topology on \textmathbb{X}, which is called the order topology.
\end{definition}

\begin{definition}[ray]\label{def:Ray}\footnote{
      open rays are always open sets in the order topology
}\footnote{
      the open rays also formed a subbasis of the order topology
}
      If \textmathbb{X} is an ordered set, and $ a $ is an element of \textmathbb{X}, there are four subsets of \textmathbb{X} that are called \defineNewWorld{rays} determined by $ a $:
      \begin{itemize}
            \item $ (a,+\infty) = \{ x | x > a \} $
            \item $ (-\infty,a) = \{ x | x < a \} $
            \item $ [a,+\infty) = \{ x | x \ge a \} $
            \item $ (-\infty,a] = \{ x | x \le a \} $
      \end{itemize}

      $ (a,+\infty) $ and $ (-\infty,a) $ are called \defineNewWorld{open rays}.
      $ [a,+\infty) $ and $ (-\infty,a] $ are called \defineNewWorld{closed rays}.
\end{definition}
      \subsection{Exercise}

      \begin{enumerate}
            \item Show that if \textmathbb{A} is a basis for a topology on \textmathbb{X}, then the topology generated by \textmathbb{A} equals the intersection of all topologies on \textmathbb{X} that contain \textmathbb{A}. Prove the same if \textmathbb{A} is a subbasis.
            
            \begin{proof}
                  As a subbasis is also a basis, we will directly prove the case of subbasis here.

                  Let $ \mathbb{S} = \{ \mathbb{T}_{\alpha} \} $ be set contain all the topologies that contain \textmathbb{A}.
                  Let \textmathbb{T} be the topology that \textmathbb{A} generated.
                  Let $ \displaystyle \mathbb{T}' = \cap\mathbb{T}_{\alpha} $.\footnote{
                        It is obvious that \textmathbb{T'} is also a topology, we just omit the proof here.
                  }

                  First, $ \mathbb{A} \subseteq \mathbb{T}_{\alpha} $.
                  Thus, $ \mathbb{T} \subseteq \mathbb{T}_{\alpha} $.
                  Thus, $ \mathbb{T} \subseteq \mathbb{T}' $.

                  Also, $ \mathbb{A} \subseteq \mathbb{T} $.
                  Thus, $ \mathbb{T} \in \mathbb{S} $.
                  Thus, $ \mathbb{T}' \subseteq \mathbb{T} $.

                  Thus, $ \mathbb{T} = \mathbb{T}' $
            \end{proof}


      \end{enumerate}

      \section{The Order Topology}

\begin{definition}[interval]\label{def:Interval}
      Let \textmathbb{X} is a set having a simple order relation $ < $. Given elements $ a $ and $ b $ of \textmathbb{X} such that $ a < b $, there are four subsets of \textmathbb{X} that are called \defineNewWorld{intervals} determined by $ a $ and $ b $:
      \begin{itemize}
            \item $ (a,b) = \{ x | a < x < b \} $
            \item $ (a,b] = \{ x | a < x \le b \} $
            \item $ [a,b) = \{ x | a \le x < b \} $
            \item $ [a,b] = \{ x | a \le x \le b \} $
      \end{itemize}

      $ (a,b) $ is called an \defineNewWorld{open interval} on \textmathbb{X}.
      $ [a,b] $ is called an \defineNewWorld{closed interval} on \textmathbb{X}.
      $ (a,b] $ and $ [a,b) $ is called \defineNewWorld{half-open intervals}.
\end{definition}

\begin{definition}[order topology]\label{def:OrderTopology}\footnote{
      The standard topology on \textmathbb{R} is an order topology derived from the usual order on \textmathbb{R}.
}
      Let \textmathbb{X} be a set with a simple order relation; assume \textmathbb{X} has more than one element. Let \textmathbb{B} be the collection of all sets of the following types:
      \begin{itemize}
            \item All open intervals $ (a,b) $ in \textmathbb{X}.
            \item All intervals of the form $ [a_{0},b) $, where $ a_0 $ is the smallest element(if exist) of \textmathbb{X}.
            \item All intervals of the form $ (a,b_{0}] $, where $ b_0 $ is the largest element(if exist) of \textmathbb{X}.
      \end{itemize}

      The collection \textmathbb{B} formed a basis for a topology on \textmathbb{X}, which is called the order topology.
\end{definition}

\begin{definition}[ray]\label{def:Ray}\footnote{
      open rays are always open sets in the order topology
}\footnote{
      the open rays also formed a subbasis of the order topology
}
      If \textmathbb{X} is an ordered set, and $ a $ is an element of \textmathbb{X}, there are four subsets of \textmathbb{X} that are called \defineNewWorld{rays} determined by $ a $:
      \begin{itemize}
            \item $ (a,+\infty) = \{ x | x > a \} $
            \item $ (-\infty,a) = \{ x | x < a \} $
            \item $ [a,+\infty) = \{ x | x \ge a \} $
            \item $ (-\infty,a] = \{ x | x \le a \} $
      \end{itemize}

      $ (a,+\infty) $ and $ (-\infty,a) $ are called \defineNewWorld{open rays}.
      $ [a,+\infty) $ and $ (-\infty,a] $ are called \defineNewWorld{closed rays}.
\end{definition}
      \subsection{Exercise}

      \begin{enumerate}
            \item Show that if \textmathbb{A} is a basis for a topology on \textmathbb{X}, then the topology generated by \textmathbb{A} equals the intersection of all topologies on \textmathbb{X} that contain \textmathbb{A}. Prove the same if \textmathbb{A} is a subbasis.
            
            \begin{proof}
                  As a subbasis is also a basis, we will directly prove the case of subbasis here.

                  Let $ \mathbb{S} = \{ \mathbb{T}_{\alpha} \} $ be set contain all the topologies that contain \textmathbb{A}.
                  Let \textmathbb{T} be the topology that \textmathbb{A} generated.
                  Let $ \displaystyle \mathbb{T}' = \cap\mathbb{T}_{\alpha} $.\footnote{
                        It is obvious that \textmathbb{T'} is also a topology, we just omit the proof here.
                  }

                  First, $ \mathbb{A} \subseteq \mathbb{T}_{\alpha} $.
                  Thus, $ \mathbb{T} \subseteq \mathbb{T}_{\alpha} $.
                  Thus, $ \mathbb{T} \subseteq \mathbb{T}' $.

                  Also, $ \mathbb{A} \subseteq \mathbb{T} $.
                  Thus, $ \mathbb{T} \in \mathbb{S} $.
                  Thus, $ \mathbb{T}' \subseteq \mathbb{T} $.

                  Thus, $ \mathbb{T} = \mathbb{T}' $
            \end{proof}


      \end{enumerate}

      \section{The Order Topology}

\begin{definition}[interval]\label{def:Interval}
      Let \textmathbb{X} is a set having a simple order relation $ < $. Given elements $ a $ and $ b $ of \textmathbb{X} such that $ a < b $, there are four subsets of \textmathbb{X} that are called \defineNewWorld{intervals} determined by $ a $ and $ b $:
      \begin{itemize}
            \item $ (a,b) = \{ x | a < x < b \} $
            \item $ (a,b] = \{ x | a < x \le b \} $
            \item $ [a,b) = \{ x | a \le x < b \} $
            \item $ [a,b] = \{ x | a \le x \le b \} $
      \end{itemize}

      $ (a,b) $ is called an \defineNewWorld{open interval} on \textmathbb{X}.
      $ [a,b] $ is called an \defineNewWorld{closed interval} on \textmathbb{X}.
      $ (a,b] $ and $ [a,b) $ is called \defineNewWorld{half-open intervals}.
\end{definition}

\begin{definition}[order topology]\label{def:OrderTopology}\footnote{
      The standard topology on \textmathbb{R} is an order topology derived from the usual order on \textmathbb{R}.
}
      Let \textmathbb{X} be a set with a simple order relation; assume \textmathbb{X} has more than one element. Let \textmathbb{B} be the collection of all sets of the following types:
      \begin{itemize}
            \item All open intervals $ (a,b) $ in \textmathbb{X}.
            \item All intervals of the form $ [a_{0},b) $, where $ a_0 $ is the smallest element(if exist) of \textmathbb{X}.
            \item All intervals of the form $ (a,b_{0}] $, where $ b_0 $ is the largest element(if exist) of \textmathbb{X}.
      \end{itemize}

      The collection \textmathbb{B} formed a basis for a topology on \textmathbb{X}, which is called the order topology.
\end{definition}

\begin{definition}[ray]\label{def:Ray}\footnote{
      open rays are always open sets in the order topology
}\footnote{
      the open rays also formed a subbasis of the order topology
}
      If \textmathbb{X} is an ordered set, and $ a $ is an element of \textmathbb{X}, there are four subsets of \textmathbb{X} that are called \defineNewWorld{rays} determined by $ a $:
      \begin{itemize}
            \item $ (a,+\infty) = \{ x | x > a \} $
            \item $ (-\infty,a) = \{ x | x < a \} $
            \item $ [a,+\infty) = \{ x | x \ge a \} $
            \item $ (-\infty,a] = \{ x | x \le a \} $
      \end{itemize}

      $ (a,+\infty) $ and $ (-\infty,a) $ are called \defineNewWorld{open rays}.
      $ [a,+\infty) $ and $ (-\infty,a] $ are called \defineNewWorld{closed rays}.
\end{definition}
      \subsection{Exercise}

      \begin{enumerate}
            \item Show that if \textmathbb{A} is a basis for a topology on \textmathbb{X}, then the topology generated by \textmathbb{A} equals the intersection of all topologies on \textmathbb{X} that contain \textmathbb{A}. Prove the same if \textmathbb{A} is a subbasis.
            
            \begin{proof}
                  As a subbasis is also a basis, we will directly prove the case of subbasis here.

                  Let $ \mathbb{S} = \{ \mathbb{T}_{\alpha} \} $ be set contain all the topologies that contain \textmathbb{A}.
                  Let \textmathbb{T} be the topology that \textmathbb{A} generated.
                  Let $ \displaystyle \mathbb{T}' = \cap\mathbb{T}_{\alpha} $.\footnote{
                        It is obvious that \textmathbb{T'} is also a topology, we just omit the proof here.
                  }

                  First, $ \mathbb{A} \subseteq \mathbb{T}_{\alpha} $.
                  Thus, $ \mathbb{T} \subseteq \mathbb{T}_{\alpha} $.
                  Thus, $ \mathbb{T} \subseteq \mathbb{T}' $.

                  Also, $ \mathbb{A} \subseteq \mathbb{T} $.
                  Thus, $ \mathbb{T} \in \mathbb{S} $.
                  Thus, $ \mathbb{T}' \subseteq \mathbb{T} $.

                  Thus, $ \mathbb{T} = \mathbb{T}' $
            \end{proof}


      \end{enumerate}

      \section{The Order Topology}

\begin{definition}[interval]\label{def:Interval}
      Let \textmathbb{X} is a set having a simple order relation $ < $. Given elements $ a $ and $ b $ of \textmathbb{X} such that $ a < b $, there are four subsets of \textmathbb{X} that are called \defineNewWorld{intervals} determined by $ a $ and $ b $:
      \begin{itemize}
            \item $ (a,b) = \{ x | a < x < b \} $
            \item $ (a,b] = \{ x | a < x \le b \} $
            \item $ [a,b) = \{ x | a \le x < b \} $
            \item $ [a,b] = \{ x | a \le x \le b \} $
      \end{itemize}

      $ (a,b) $ is called an \defineNewWorld{open interval} on \textmathbb{X}.
      $ [a,b] $ is called an \defineNewWorld{closed interval} on \textmathbb{X}.
      $ (a,b] $ and $ [a,b) $ is called \defineNewWorld{half-open intervals}.
\end{definition}

\begin{definition}[order topology]\label{def:OrderTopology}\footnote{
      The standard topology on \textmathbb{R} is an order topology derived from the usual order on \textmathbb{R}.
}
      Let \textmathbb{X} be a set with a simple order relation; assume \textmathbb{X} has more than one element. Let \textmathbb{B} be the collection of all sets of the following types:
      \begin{itemize}
            \item All open intervals $ (a,b) $ in \textmathbb{X}.
            \item All intervals of the form $ [a_{0},b) $, where $ a_0 $ is the smallest element(if exist) of \textmathbb{X}.
            \item All intervals of the form $ (a,b_{0}] $, where $ b_0 $ is the largest element(if exist) of \textmathbb{X}.
      \end{itemize}

      The collection \textmathbb{B} formed a basis for a topology on \textmathbb{X}, which is called the order topology.
\end{definition}

\begin{definition}[ray]\label{def:Ray}\footnote{
      open rays are always open sets in the order topology
}\footnote{
      the open rays also formed a subbasis of the order topology
}
      If \textmathbb{X} is an ordered set, and $ a $ is an element of \textmathbb{X}, there are four subsets of \textmathbb{X} that are called \defineNewWorld{rays} determined by $ a $:
      \begin{itemize}
            \item $ (a,+\infty) = \{ x | x > a \} $
            \item $ (-\infty,a) = \{ x | x < a \} $
            \item $ [a,+\infty) = \{ x | x \ge a \} $
            \item $ (-\infty,a] = \{ x | x \le a \} $
      \end{itemize}

      $ (a,+\infty) $ and $ (-\infty,a) $ are called \defineNewWorld{open rays}.
      $ [a,+\infty) $ and $ (-\infty,a] $ are called \defineNewWorld{closed rays}.
\end{definition}
      \subsection{Exercise}

      \begin{enumerate}
            \item Show that if \textmathbb{A} is a basis for a topology on \textmathbb{X}, then the topology generated by \textmathbb{A} equals the intersection of all topologies on \textmathbb{X} that contain \textmathbb{A}. Prove the same if \textmathbb{A} is a subbasis.
            
            \begin{proof}
                  As a subbasis is also a basis, we will directly prove the case of subbasis here.

                  Let $ \mathbb{S} = \{ \mathbb{T}_{\alpha} \} $ be set contain all the topologies that contain \textmathbb{A}.
                  Let \textmathbb{T} be the topology that \textmathbb{A} generated.
                  Let $ \displaystyle \mathbb{T}' = \cap\mathbb{T}_{\alpha} $.\footnote{
                        It is obvious that \textmathbb{T'} is also a topology, we just omit the proof here.
                  }

                  First, $ \mathbb{A} \subseteq \mathbb{T}_{\alpha} $.
                  Thus, $ \mathbb{T} \subseteq \mathbb{T}_{\alpha} $.
                  Thus, $ \mathbb{T} \subseteq \mathbb{T}' $.

                  Also, $ \mathbb{A} \subseteq \mathbb{T} $.
                  Thus, $ \mathbb{T} \in \mathbb{S} $.
                  Thus, $ \mathbb{T}' \subseteq \mathbb{T} $.

                  Thus, $ \mathbb{T} = \mathbb{T}' $
            \end{proof}


      \end{enumerate}

      \section{The Order Topology}

\begin{definition}[interval]\label{def:Interval}
      Let \textmathbb{X} is a set having a simple order relation $ < $. Given elements $ a $ and $ b $ of \textmathbb{X} such that $ a < b $, there are four subsets of \textmathbb{X} that are called \defineNewWorld{intervals} determined by $ a $ and $ b $:
      \begin{itemize}
            \item $ (a,b) = \{ x | a < x < b \} $
            \item $ (a,b] = \{ x | a < x \le b \} $
            \item $ [a,b) = \{ x | a \le x < b \} $
            \item $ [a,b] = \{ x | a \le x \le b \} $
      \end{itemize}

      $ (a,b) $ is called an \defineNewWorld{open interval} on \textmathbb{X}.
      $ [a,b] $ is called an \defineNewWorld{closed interval} on \textmathbb{X}.
      $ (a,b] $ and $ [a,b) $ is called \defineNewWorld{half-open intervals}.
\end{definition}

\begin{definition}[order topology]\label{def:OrderTopology}\footnote{
      The standard topology on \textmathbb{R} is an order topology derived from the usual order on \textmathbb{R}.
}
      Let \textmathbb{X} be a set with a simple order relation; assume \textmathbb{X} has more than one element. Let \textmathbb{B} be the collection of all sets of the following types:
      \begin{itemize}
            \item All open intervals $ (a,b) $ in \textmathbb{X}.
            \item All intervals of the form $ [a_{0},b) $, where $ a_0 $ is the smallest element(if exist) of \textmathbb{X}.
            \item All intervals of the form $ (a,b_{0}] $, where $ b_0 $ is the largest element(if exist) of \textmathbb{X}.
      \end{itemize}

      The collection \textmathbb{B} formed a basis for a topology on \textmathbb{X}, which is called the order topology.
\end{definition}

\begin{definition}[ray]\label{def:Ray}\footnote{
      open rays are always open sets in the order topology
}\footnote{
      the open rays also formed a subbasis of the order topology
}
      If \textmathbb{X} is an ordered set, and $ a $ is an element of \textmathbb{X}, there are four subsets of \textmathbb{X} that are called \defineNewWorld{rays} determined by $ a $:
      \begin{itemize}
            \item $ (a,+\infty) = \{ x | x > a \} $
            \item $ (-\infty,a) = \{ x | x < a \} $
            \item $ [a,+\infty) = \{ x | x \ge a \} $
            \item $ (-\infty,a] = \{ x | x \le a \} $
      \end{itemize}

      $ (a,+\infty) $ and $ (-\infty,a) $ are called \defineNewWorld{open rays}.
      $ [a,+\infty) $ and $ (-\infty,a] $ are called \defineNewWorld{closed rays}.
\end{definition}
      \subsection{Exercise}

      \begin{enumerate}
            \item Show that if \textmathbb{A} is a basis for a topology on \textmathbb{X}, then the topology generated by \textmathbb{A} equals the intersection of all topologies on \textmathbb{X} that contain \textmathbb{A}. Prove the same if \textmathbb{A} is a subbasis.
            
            \begin{proof}
                  As a subbasis is also a basis, we will directly prove the case of subbasis here.

                  Let $ \mathbb{S} = \{ \mathbb{T}_{\alpha} \} $ be set contain all the topologies that contain \textmathbb{A}.
                  Let \textmathbb{T} be the topology that \textmathbb{A} generated.
                  Let $ \displaystyle \mathbb{T}' = \cap\mathbb{T}_{\alpha} $.\footnote{
                        It is obvious that \textmathbb{T'} is also a topology, we just omit the proof here.
                  }

                  First, $ \mathbb{A} \subseteq \mathbb{T}_{\alpha} $.
                  Thus, $ \mathbb{T} \subseteq \mathbb{T}_{\alpha} $.
                  Thus, $ \mathbb{T} \subseteq \mathbb{T}' $.

                  Also, $ \mathbb{A} \subseteq \mathbb{T} $.
                  Thus, $ \mathbb{T} \in \mathbb{S} $.
                  Thus, $ \mathbb{T}' \subseteq \mathbb{T} $.

                  Thus, $ \mathbb{T} = \mathbb{T}' $
            \end{proof}


      \end{enumerate}

      \section{The Order Topology}

\begin{definition}[interval]\label{def:Interval}
      Let \textmathbb{X} is a set having a simple order relation $ < $. Given elements $ a $ and $ b $ of \textmathbb{X} such that $ a < b $, there are four subsets of \textmathbb{X} that are called \defineNewWorld{intervals} determined by $ a $ and $ b $:
      \begin{itemize}
            \item $ (a,b) = \{ x | a < x < b \} $
            \item $ (a,b] = \{ x | a < x \le b \} $
            \item $ [a,b) = \{ x | a \le x < b \} $
            \item $ [a,b] = \{ x | a \le x \le b \} $
      \end{itemize}

      $ (a,b) $ is called an \defineNewWorld{open interval} on \textmathbb{X}.
      $ [a,b] $ is called an \defineNewWorld{closed interval} on \textmathbb{X}.
      $ (a,b] $ and $ [a,b) $ is called \defineNewWorld{half-open intervals}.
\end{definition}

\begin{definition}[order topology]\label{def:OrderTopology}\footnote{
      The standard topology on \textmathbb{R} is an order topology derived from the usual order on \textmathbb{R}.
}
      Let \textmathbb{X} be a set with a simple order relation; assume \textmathbb{X} has more than one element. Let \textmathbb{B} be the collection of all sets of the following types:
      \begin{itemize}
            \item All open intervals $ (a,b) $ in \textmathbb{X}.
            \item All intervals of the form $ [a_{0},b) $, where $ a_0 $ is the smallest element(if exist) of \textmathbb{X}.
            \item All intervals of the form $ (a,b_{0}] $, where $ b_0 $ is the largest element(if exist) of \textmathbb{X}.
      \end{itemize}

      The collection \textmathbb{B} formed a basis for a topology on \textmathbb{X}, which is called the order topology.
\end{definition}

\begin{definition}[ray]\label{def:Ray}\footnote{
      open rays are always open sets in the order topology
}\footnote{
      the open rays also formed a subbasis of the order topology
}
      If \textmathbb{X} is an ordered set, and $ a $ is an element of \textmathbb{X}, there are four subsets of \textmathbb{X} that are called \defineNewWorld{rays} determined by $ a $:
      \begin{itemize}
            \item $ (a,+\infty) = \{ x | x > a \} $
            \item $ (-\infty,a) = \{ x | x < a \} $
            \item $ [a,+\infty) = \{ x | x \ge a \} $
            \item $ (-\infty,a] = \{ x | x \le a \} $
      \end{itemize}

      $ (a,+\infty) $ and $ (-\infty,a) $ are called \defineNewWorld{open rays}.
      $ [a,+\infty) $ and $ (-\infty,a] $ are called \defineNewWorld{closed rays}.
\end{definition}
      \subsection{Exercise}

      \begin{enumerate}
            \item Show that if \textmathbb{A} is a basis for a topology on \textmathbb{X}, then the topology generated by \textmathbb{A} equals the intersection of all topologies on \textmathbb{X} that contain \textmathbb{A}. Prove the same if \textmathbb{A} is a subbasis.
            
            \begin{proof}
                  As a subbasis is also a basis, we will directly prove the case of subbasis here.

                  Let $ \mathbb{S} = \{ \mathbb{T}_{\alpha} \} $ be set contain all the topologies that contain \textmathbb{A}.
                  Let \textmathbb{T} be the topology that \textmathbb{A} generated.
                  Let $ \displaystyle \mathbb{T}' = \cap\mathbb{T}_{\alpha} $.\footnote{
                        It is obvious that \textmathbb{T'} is also a topology, we just omit the proof here.
                  }

                  First, $ \mathbb{A} \subseteq \mathbb{T}_{\alpha} $.
                  Thus, $ \mathbb{T} \subseteq \mathbb{T}_{\alpha} $.
                  Thus, $ \mathbb{T} \subseteq \mathbb{T}' $.

                  Also, $ \mathbb{A} \subseteq \mathbb{T} $.
                  Thus, $ \mathbb{T} \in \mathbb{S} $.
                  Thus, $ \mathbb{T}' \subseteq \mathbb{T} $.

                  Thus, $ \mathbb{T} = \mathbb{T}' $
            \end{proof}


      \end{enumerate}

      \section{The Order Topology}

\begin{definition}[interval]\label{def:Interval}
      Let \textmathbb{X} is a set having a simple order relation $ < $. Given elements $ a $ and $ b $ of \textmathbb{X} such that $ a < b $, there are four subsets of \textmathbb{X} that are called \defineNewWorld{intervals} determined by $ a $ and $ b $:
      \begin{itemize}
            \item $ (a,b) = \{ x | a < x < b \} $
            \item $ (a,b] = \{ x | a < x \le b \} $
            \item $ [a,b) = \{ x | a \le x < b \} $
            \item $ [a,b] = \{ x | a \le x \le b \} $
      \end{itemize}

      $ (a,b) $ is called an \defineNewWorld{open interval} on \textmathbb{X}.
      $ [a,b] $ is called an \defineNewWorld{closed interval} on \textmathbb{X}.
      $ (a,b] $ and $ [a,b) $ is called \defineNewWorld{half-open intervals}.
\end{definition}

\begin{definition}[order topology]\label{def:OrderTopology}\footnote{
      The standard topology on \textmathbb{R} is an order topology derived from the usual order on \textmathbb{R}.
}
      Let \textmathbb{X} be a set with a simple order relation; assume \textmathbb{X} has more than one element. Let \textmathbb{B} be the collection of all sets of the following types:
      \begin{itemize}
            \item All open intervals $ (a,b) $ in \textmathbb{X}.
            \item All intervals of the form $ [a_{0},b) $, where $ a_0 $ is the smallest element(if exist) of \textmathbb{X}.
            \item All intervals of the form $ (a,b_{0}] $, where $ b_0 $ is the largest element(if exist) of \textmathbb{X}.
      \end{itemize}

      The collection \textmathbb{B} formed a basis for a topology on \textmathbb{X}, which is called the order topology.
\end{definition}

\begin{definition}[ray]\label{def:Ray}\footnote{
      open rays are always open sets in the order topology
}\footnote{
      the open rays also formed a subbasis of the order topology
}
      If \textmathbb{X} is an ordered set, and $ a $ is an element of \textmathbb{X}, there are four subsets of \textmathbb{X} that are called \defineNewWorld{rays} determined by $ a $:
      \begin{itemize}
            \item $ (a,+\infty) = \{ x | x > a \} $
            \item $ (-\infty,a) = \{ x | x < a \} $
            \item $ [a,+\infty) = \{ x | x \ge a \} $
            \item $ (-\infty,a] = \{ x | x \le a \} $
      \end{itemize}

      $ (a,+\infty) $ and $ (-\infty,a) $ are called \defineNewWorld{open rays}.
      $ [a,+\infty) $ and $ (-\infty,a] $ are called \defineNewWorld{closed rays}.
\end{definition}
      \subsection{Exercise}

      \begin{enumerate}
            \item Show that if \textmathbb{A} is a basis for a topology on \textmathbb{X}, then the topology generated by \textmathbb{A} equals the intersection of all topologies on \textmathbb{X} that contain \textmathbb{A}. Prove the same if \textmathbb{A} is a subbasis.
            
            \begin{proof}
                  As a subbasis is also a basis, we will directly prove the case of subbasis here.

                  Let $ \mathbb{S} = \{ \mathbb{T}_{\alpha} \} $ be set contain all the topologies that contain \textmathbb{A}.
                  Let \textmathbb{T} be the topology that \textmathbb{A} generated.
                  Let $ \displaystyle \mathbb{T}' = \cap\mathbb{T}_{\alpha} $.\footnote{
                        It is obvious that \textmathbb{T'} is also a topology, we just omit the proof here.
                  }

                  First, $ \mathbb{A} \subseteq \mathbb{T}_{\alpha} $.
                  Thus, $ \mathbb{T} \subseteq \mathbb{T}_{\alpha} $.
                  Thus, $ \mathbb{T} \subseteq \mathbb{T}' $.

                  Also, $ \mathbb{A} \subseteq \mathbb{T} $.
                  Thus, $ \mathbb{T} \in \mathbb{S} $.
                  Thus, $ \mathbb{T}' \subseteq \mathbb{T} $.

                  Thus, $ \mathbb{T} = \mathbb{T}' $
            \end{proof}


      \end{enumerate}

\end{document}